% Created 2016-06-02 Thu 16:16
\documentclass[aspectratio=169]{beamer}
\usepackage[utf8]{inputenc}
\usepackage[T1]{fontenc}
\usepackage{amsmath}
\usepackage{booktabs}
\usepackage{sourcecodepro}
%\usepackage{sourcesanspro}
\usepackage{arev}
\usepackage{tikz}
\usetikzlibrary{arrows,shapes,positioning}
\usepackage{dot2texi}
%\usepackage{tikz-uml}
%\usepackage[pdf]{graphviz}
\usepackage{minted}

\usetheme{Goettingen}
\usecolortheme{}
\usefonttheme[onlysmall]{structurebold}
\useinnertheme{}
\useoutertheme{}
\setbeamertemplate{navigation symbols}{}

\date{Auckland C++ Meetup\\19 July 2016}
\title{Type-safe Unions in C++14}
\subtitle{\small\url{https://github.com/toby-allsopp/auck_cpp-typesafe_unions}}
\author[Toby Allsopp]{Toby Allsopp\\\texttt{toby@mi6.gen.nz}}

\hypersetup{
 pdfauthor={},
 pdftitle={},
 pdfkeywords={},
 pdfsubject={},
 pdfcreator={},
 pdflang={English}}

\begin{document}

%\usemintedstyle{murphy}
\setminted{frame=single}
\newminted{cpp}{autogobble}
\newmintinline[cpp]{cpp}{}

\frame{\titlepage}

\section{Introduction}

\begin{frame}
  \frametitle{Notes}
  \begin{itemize}
  \item There is lots of code in this talk
  \item All code is on github
  \item Intended to conform to C++14
  \item Tested with GCC 6.1.1 and clang 3.8.0
  \item Might work with VS2015
  \item I'm not an expert on TMP; this code is almost certainly inefficient
    \begin{itemize}
    \item at coding time
    \item at compile time
    \item at run time
    \end{itemize}
  \item Feel free to shout out suggestions
  \end{itemize}
\end{frame}

\begin{frame}
  \frametitle{Overview}
  \tableofcontents
\end{frame}


\section{Motivating example}

\begin{frame}[fragile]
  \frametitle{A robot}  
  %\digraph[scale=0.5]{abc}{
  \begin{center}
  \begin{dot2tex}[dot,autosize,scale=0.6]
    digraph {
      center=1;
      rankdir=LR;
      node [shape=box];
      start [shape=circle,style=filled,fillcolor=black,label=""];
      decision [shape=diamond, label=""];
      start->off [weight=10];
      idle->turning [label="start_turning"];
      turning->decision [weight=2,label="heading_changed"];
      decision->turning [label="[heading!=target]"];
      decision->idle [label="[heading==target]"];
      turning->idle [label="reset"];
      off->idle [label="turn_on"];
      idle->off [label="turn_off"];
      turning->off [label="turn_off"];
    }
  \end{dot2tex}
  \end{center}
  \begin{itemize}
  \item States
\begin{cppcode*}{fontsize=\footnotesize}
struct off {};
struct idle {};
struct turning { float target; };
\end{cppcode*}
  \item Events
\begin{cppcode*}{fontsize=\footnotesize}
struct turn_on {};
struct turn_off {};
struct start_turning { float target; };
struct heading_changed { float heading; };
struct reset { std::string reason; };
\end{cppcode*}
  \end{itemize}
\end{frame}

\begin{frame}[fragile]
  \begin{itemize}
  \item We want to write a state transition function for the robot
\begin{cppcode}
state transition(const state& s, const event& e);
\end{cppcode}
  \item But what types do we use for \cpp|state| and \cpp|event|?
  \item We need something that can hold \alert{either} an \cpp|off|, an
    \cpp|idle| or a \cpp|turning| (and similarly for the events)
  \item Something like a \cpp|union|
  \item But better!
  \end{itemize}
\end{frame}


\section{Discriminated unions}

\begin{frame}[fragile]
  \frametitle{\cpp|variant|}
  \begin{itemize}
  \item Boost has \cpp|boost::variant|
  \item C++17 will have \cpp|std::variant|
  \item We're going to build our own
\begin{cppcode}
using state = variant<off, idle, turning>;
using event =
    variant<turn_on, turn_off, start_turning, reset,
            heading_changed>;
\end{cppcode}
  \end{itemize}
\end{frame}

\begin{frame}[fragile]
  \frametitle{Indiscriminate \cpp|union|}
\begin{cppcode}
    union u {
      int i;
      double d;
    }
    u x;
    x.i = 3;
    double d = x.d; // UNDEFINED BEHAVIOUR
\end{cppcode}
  \begin{itemize}
  \item Can't tell what was last stored but you'd better know!
  \item Can be used for bit twiddling on specific platforms
  \item No guarantees from the standard
  \end{itemize}
\end{frame}

\begin{frame}[fragile]
  \frametitle{More \cpp|union| gotchas}
\begin{cppcode}
union u {
  std::string s;
  std::vector<int> v;
} x;
x.s = "Hello"; // KABOOM!
new (&x.s) std::string("Hello"); // OK
x.s = "Goodbye"; // OK
u y = x; // NOPE - copy constructor is deleted
new (&x.v) std::vector<int>{1, 2, 3}; // LEAK!
x.s.~std::string(); // DO THIS FIRST
\end{cppcode}
\end{frame}

\begin{frame}[fragile]
  \frametitle{\cpp|union| + \cpp|class| + \cpp|enum|}
\begin{cppcode}
class svu {
 private:
  enum { STRING, VECTOR } tag;
  union {
    std::string s;
    std::vector<int> v;
  };
};
\end{cppcode}
  \begin{itemize}
  \item Because it's so unsafe, let's bundle it up into a class and only expose
    safe operations.
  \item We need to keep track of which union member we have initialized - we
    call this a \alert{tag}.
  \end{itemize}
\end{frame}

\begin{frame}[fragile]
  \frametitle{Construction}
\begin{cppcode}
 private:
  void construct(const std::string& _s) {
    tag = STRING;
    new (&s) std::string(_s);
  }
  void construct(const std::vector<int>& _v) {
    tag = VECTOR;
    new (&v) std::vector<int>(_v);
  }
 public:
  template <typename T>
  svu(const T& x) {
    construct(x);
  }
\end{cppcode}
\end{frame}

\begin{frame}[fragile]
  \frametitle{Visitation}
\begin{cppcode}
 public:
  template <typename R, typename F>
  R visit(F&& f) {
    switch (tag) {
      case STRING: return f(s);
      case VECTOR: return f(v);
    }
  }
  // and a const version
\end{cppcode}
  \begin{itemize}
  \item So the object you pass in has to have an \cpp|operator()| that takes a
    \cpp|string| and one that takes a \cpp|vector<int>|.
  \item Both operators must return something convertible to \cpp|R|.
  \end{itemize}
\end{frame}

\begin{frame}[fragile]
  \frametitle{Visitation example}
\begin{cppcode}
struct Visitor {
  int operator()(const string& s) { return s.size(); }
  int operator()(const vector<int> v) { return v[0]; }
};
svu x("Hello");
int i = x.visit<int>(Visitor()); // 5
int i = x.visit<int>([](const auto& v) {
                       return v.size();
                     });
\end{cppcode}
\end{frame} 

\begin{frame}[fragile]
  \frametitle{Destruction}
\begin{cppcode}
 private:
  void destruct() {
    visit<void>([](auto&& x) {
      using T = std::decay_t<decltype(x)>;
      x.~T();
    });
  }
 public:
  ~svu() { destruct(); }
\end{cppcode}
  \begin{itemize}
  \item \cpp|decay_t|?
  \item This has no dependencies on the particular types we're using.
  \end{itemize}
\end{frame}

\begin{frame}[fragile]
  \frametitle{Assignment}
\begin{cppcode}
  svu& operator=(const svu& other) {
    destruct();
    other.visit<void>(
        [this](auto&& v) { construct(v); });
    return *this;
  }
\end{cppcode}
\end{frame}

\begin{frame}
  \frametitle{Move semantics}
  \begin{itemize}
  \item We can define move copy constructors and assignment operators
  \item And we should for efficiency's sake
  \item But it's tedious --- just chuck some \cpp|&&|s and \cpp|std::move|s around
    the place
  \end{itemize}
\end{frame}


\section{Inline visitors}

\begin{frame}[fragile,fragile]
  \frametitle{Inline visitors}
  \begin{itemize}
  \item Recall how we defined a visitor earlier:
\begin{cppcode}
struct Visitor {
  int operator()(const string& s) {return s.size();}
  int operator()(const vector<int> v) {return v[0];}
};
\end{cppcode}
  \item Wouldn't it be nice to not have to explicitly define a struct for this?
\begin{cppcode}
x.visit<int>(
    [](const string& s) { return s.size(); },
    [](const vector<int> v) { return v[0]; });
\end{cppcode}
  \item We can do this with a little recursive class template...
  \end{itemize}
\end{frame}
  
\begin{frame}[fragile]
  \frametitle{Overload set}
  \begin{itemize}
  \item We start by declaring the class template. The template parameters are
    the types of the function-like objects that implement it:
\begin{cppcode}
template <typename... Fs>
class overload_set;
\end{cppcode}

  \item Then we define the base case for the recursion - an overload set with
    zero functions:
\begin{cppcode}
template <>
class overload_set<> {
 public:
  void operator()() = delete;
};
\end{cppcode}
  \end{itemize}
\end{frame}

\begin{frame}[fragile]
  \frametitle{Overload set}
  \begin{itemize}
  \item Finally we define the inductive case --- an overload set with $n+1$
    functions defined in terms of one with $n$:
  \end{itemize}
\begin{cppcode}
template <typename F, typename... Fs>
class overload_set<F, Fs...>
    : private overload_set<Fs...>, private F {
 public:
  explicit overload_set(F&& f, Fs&&... fs)
      : overload_set<Fs...>(std::forward<Fs>(fs)...),
        F(std::forward<F>(f)) {}

  using F::operator();
  using overload_set<Fs...>::operator();
};
\end{cppcode}
\end{frame}

\begin{frame}[fragile]
  \begin{itemize}
  \item Now we can add an overload of our \cpp|visit| function to create an
    \cpp|overload_set| if we pass other than one argument.
\begin{cppcode}
  template <typename R, typename... Fs>
  R visit(Fs&&... fs) {
    return visit<R>(
        overload_set<Fs...>(std::forward<Fs>(fs)...));
  }
  // and a const version
\end{cppcode}
  \end{itemize}
\end{frame}

\begin{frame}[fragile]
  \frametitle{Inline visitation}
\begin{cppcode}
  x.visit<int>(
      [](const std::string& s) { return s.size(); },
      [](const std::vector<int> v) { return v[0]; });
\end{cppcode}
  \begin{itemize}
  \item Note how \cpp|s.size()| actually returns \cpp|size_t| but it gets
    implicitly converted to \cpp|int| --- this is useful in many cases but
    consider \verb|-Wconversion|.
  \end{itemize}
\end{frame}

\section{Multi-visitation}

\begin{frame}[fragile]
  \frametitle{Multi-visitation}
  \begin{itemize}
  \item What if we want to visit two objects at once?
\begin{cppcode*}{fontsize=\tiny}
int plux(const svu& u1, const svu& u2) {
  using namespace std;
  return u1.visit<int>(
      [&](const string& s1) {
        return u2.visit<int>(
            [&](const string& s2)      { return s1.size() + s2.size(); },
            [&](const vector<int>& v2) { return s1.size() + v2[0]; });
      },
      [&](const vector<int> v1) {
        return u2.visit<int>(
            [&](const string& s2)      { return v1.size() + s2[0]; },
            [&](const vector<int>& v2) { return v1[0] + v2[0]; });
      });
}
\end{cppcode*}
  \item That makes me want to claw my eyes out.
  \end{itemize}
\end{frame}

\begin{frame}[fragile]
  \begin{itemize}
  \item I want to be able to write:
\begin{cppcode*}{fontsize=\tiny}
int plux2(const svu& u1, const svu& u2) {
  using namespace std;
  auto visitor = make_multivisitor<int>(
      [](const string& s1,     const string& s2)      { return s1.size() + s2.size(); },
      [](const string& s1,     const vector<int>& v2) { return s1.size() + v2[0]; },
      [](const vector<int> v1, const string& s2)      { return v1.size() + s2[0]; },
      [](const vector<int> v1, const vector<int>& v2) { return v1[0] + v2[0]; });
  return visitor(u1, u2);
}
\end{cppcode*}
  \item This is much nicer because
    \begin{itemize}
    \item it scales gracefully to any number of visitees,
    \item it allows wildcards in any position and
    \item it only makes we want to claw one eye out.
    \end{itemize}
  \item But how do we make one?
  \end{itemize}
\end{frame}
  
\begin{frame}[fragile]
  \begin{itemize}
  \item It starts out pretty simple...
  \end{itemize}
\begin{cppcode}
template <typename R, typename F>
class multivisitor {
 private:
  F m_f;

 public:
  explicit multivisitor(F&& f) : m_f(f) {}

  template <typename... Vs>
  auto operator()(const Vs&... args) {
    return collect(std::tuple<>(), args...);
  }
\end{cppcode}
\end{frame}

\begin{frame}[fragile]
  \begin{itemize}
  \item The tricky bit is that we need to accumulate the results of visiting
    each variant until we've visited them all.
  \end{itemize}
\begin{cppcode*}{fontsize=\footnotesize}
 private:
  template <typename T>
  auto collect(const T& t) {
    return apply(m_f, t);
  }

  template <typename T, typename V, typename... Vs>
  auto collect(const T& t, const V& arg, const Vs&... args) {
    return arg.template visit<R>([&](auto v) {
      return this->collect(
          std::tuple_cat(t, std::make_tuple(v)), args...);
    });
  }
};
\end{cppcode*}
\end{frame}

\begin{frame}[fragile]
  \begin{itemize}
  \item OK, now we have a tuple of values --- how do we pass them to our
    function?
  \item C++17 has \cpp|std::apply| but we can copy a cheap imposter from
    \href{http://en.cppreference.com/w/cpp/utility/apply}{\url{cppreference.com}}...
  \end{itemize}
\begin{cppcode*}{fontsize=\footnotesize}
template <typename Callable, typename Tuple, size_t... I>
auto apply_impl(Callable&& f, Tuple&& t,
                std::index_sequence<I...>) {
  return f(std::get<I>(t)...);
}

template <typename Callable, typename Tuple>
auto apply(Callable&& f, Tuple&& t) {
  using is = std::make_index_sequence<
      std::tuple_size<std::decay_t<Tuple>>::value>;
  return apply_impl(std::forward<Callable>(f),
                    std::forward<Tuple>(t), is());
}
\end{cppcode*}
\end{frame}


\section{Generic variant}

\begin{frame}
  \begin{itemize}
  \item So, that's all well and good if you want a variant that can hold a
    \cpp|string| or a \cpp|vector<int>|.
  \item But we wanted two different variants --- one for states and one for
    events.
  \item So let's copy and paste and change the names, job done.
  \end{itemize}
\end{frame}

\begin{frame}
  \begin{center}
    \Huge{Hell, no!}
  \end{center}
  \begin{center}
    We're just getting warmed up.
  \end{center}
\end{frame}

\begin{frame}[fragile]
  \frametitle{Construction}
  \begin{itemize}
  \item We need a \cpp|construct| function for each type in the variant.
  \item So we go to our old friend, the recursive class template.
\begin{cppcode}
template <typename I, I N, typename... Ts>
struct variant_construct;
\end{cppcode}
  \item The base case, for a variant with zero types:
\begin{cppcode}
template <typename I, I N>
struct variant_construct<I, N> {
  static I construct();
};
\end{cppcode}
  \end{itemize}
\end{frame}

\begin{frame}[fragile]
  \frametitle{Construction}
  \begin{itemize}
  \item And the inductive $n+1$ case defined in terms of the $n$ case:
  \end{itemize}
\begin{cppcode}
template <typename I, I N, typename T, typename... Ts>
struct variant_construct<I, N, T, Ts...>
    : private variant_construct<I, N + 1, Ts...> {
  using super = variant_construct<I, N + 1, Ts...>;

  static I construct(void* storage, const T& value) {
    new (storage) T(value);
    return N;
  }
  using super::construct;
};
\end{cppcode}
\end{frame}

\begin{frame}[fragile]
  \frametitle{Visitation}
\begin{cppcode*}{fontsize=\tiny}
template <typename I, I N, typename... Ts>
struct variant_visit;

template <typename I, I N>
struct variant_visit<I, N> {
  template <typename R, typename F>
  static R visit_helper_const(I tag, const void*, F&&) {
    throw std::logic_error("variant tag invalid");
  }
};

template <typename I, I N, typename T, typename... Ts>
struct variant_visit<I, N, T, Ts...> : private variant_visit<I, N + 1, Ts...> {
  using super = variant_visit<I, N + 1, Ts...>;

  template <typename R, typename F>
  static R visit_helper_const(I tag, const void* storage, F&& f) {
    if (tag == N) {
      return f(*reinterpret_cast<const T*>(storage));
    }
    return super::template visit_helper_const<R>(tag, storage, std::forward<F>(f));
  }
};
\end{cppcode*}
\end{frame}

\begin{frame}[fragile]
  \frametitle{Storage}
  \begin{itemize}
  \item I can't figure out how to use a \cpp|union| as storage.
  \item But we can use a very handy template:
\begin{cppcode}
  typename std::aligned_union_t<0, Ts...> storage;
\end{cppcode}
  \item The size needed for the tag depends on how many types are in the
    variant.
\begin{cppcode*}{fontsize=\tiny}
template <uintmax_t N, typename Enable = void>
struct smallest_unisnged_type;

template <uintmax_t N>
struct smallest_unisnged_type<
    N, typename std::enable_if_t<N <= std::numeric_limits<uint8_t>::max()>> {
  using type = uint8_t;
};

template <uintmax_t N>
using smallest_unisnged_type_t = typename smallest_unisnged_type<N>::type;
\end{cppcode*}
  \end{itemize}
\end{frame}

\begin{frame}[fragile]
  \frametitle{\cpp|variant| part 1}
\begin{cppcode*}{fontsize=\tiny}
template <typename... Ts>
struct variant_helper {
  using tag_type = smallest_unisnged_type_t<sizeof...(Ts)>;
  using super_construct = variant_construct<tag_type, 0, Ts...>;
  using super_visit = variant_visit<tag_type, 0, Ts...>;
};

template <typename... Ts>
class variant : private variant_helper<Ts...>::super_construct,
                private variant_helper<Ts...>::super_visit {
  using helper = variant_helper<Ts...>;
  using super_construct = typename helper::super_construct;
  using super_visit = typename helper::super_visit;

  typename std::aligned_union_t<0, char, Ts...> storage;
  typename helper::tag_type tag;

  using super_construct::construct;

  void destruct() {
    std::move(*this).template visit<void>([this](auto&& v) {
      using T = std::decay_t<decltype(v)>;
      reinterpret_cast<T*>(&storage)->~T();
    });
  }
\end{cppcode*}
\end{frame}

\begin{frame}[fragile]
  \frametitle{\cpp|variant| part 2}
\begin{cppcode*}{fontsize=\tiny}
 public:
  template <typename T, typename = decltype(construct(
                            &storage, std::forward<T>(std::declval<T>())))>
  variant(T&& value) {
    tag = construct(&storage, std::forward<T>(value));
  }

  variant(const variant& other) {
    other.visit<void>(
        [this](auto&& value) { tag = construct(&storage, value); });
  }

  variant& operator=(const variant& other) {
    destruct();
    other.visit<void>(
        [this](auto&& value) { tag = construct(&storage, value); });
    return *this;
  }

  ~variant() { destruct(); }

  template <typename R, typename F>
  auto visit(F&& f) const& {
    return super_visit::template visit_helper_const<R>(tag, &storage,
                                                       std::forward<F>(f));
  }
};
\end{cppcode*}
\end{frame}

\begin{frame}[fragile]
  \frametitle{Variant type}
  \begin{itemize}
  \item Finally, we can define the types for our robot's state and events
  \begin{cppcode*}{fontsize=\footnotesize}
using state = variant<off, idle, turning>;
using on = variant<idle, turning>;

using event = variant<turn_on, turn_off, start_turning, reset,
                      heading_changed>;
\end{cppcode*}
  \item That \cpp|on| type comes in handy for the state transitions.
  \end{itemize}
\end{frame}

\begin{frame}[fragile]
  \frametitle{State transitions}
  \begin{cppcode*}{fontsize=\footnotesize}
state transition(const state& s, const event& e) {
  return make_multivisitor<state>(
      [](off,       turn_on)         { return idle{}; },
      [](off,       auto)            { return off{}; },
      [](on,        turn_off)        { return off{}; },
      [](auto s,    turn_on)         { return s; },
      [](on,        reset)           { return idle{}; },
      [](on,        start_turning e) { return turning{e.target}; },
      [](idle s,    heading_changed) { return s; },
      [](turning s, heading_changed e) -> state {
        if (std::abs(e.heading - s.target) < .1f) {
          return idle{};
        } else {
          return s;
        }
      })(s, e);
}
\end{cppcode*}
\end{frame}


\section{Questions}

\begin{frame}
  \centering{Questions?}
\end{frame}


\appendix

\section{\appendixname}
\frame{\tableofcontents}

\subsection{Extra material}

\subsection{References}

\begin{frame}[fragile]
  \frametitle{References}
  \begin{itemize}
  \item Boost
  \end{itemize}
\end{frame}

\end{document}

% Local Variables:
% TeX-command-extra-options: "-shell-escape"
% End:
